\section*{Conventions and Notation}\label{sec:conventions-and-notation}

\subsection*{Hierarchical Organization}\label{subsec:hierarchical-organization}
The document is structured using a clear hierarchical system:
\begin{itemize}
    \item \textbf{Chapters} represent major topics and are numbered sequentially
    \item \textbf{Sections} divide chapters into primary subtopics
    \item \textbf{Subsections} provide further subdivision of complex topics
    \item \textbf{Subsubsections} offer detailed breakdowns when necessary
\end{itemize}

All theorems, definitions, examples, and other mathematical constructs are numbered within their respective chapters using the format \texttt{Type Chapter.Number} (e.g., Definition 2.1, Theorem 3.5).

\subsection*{Special Environments}\label{subsec:special-environments}

The document employs several specialized environments to categorize different types of content:

\begin{definition}
    Formal definitions of terms, concepts, and mathematical objects are presented in definition boxes like this one.
    Definitions establish precise meanings for terminology used throughout the document.
\end{definition}

\begin{theorem}
    Mathematical statements that have been proven are presented as theorems.
    Each theorem is numbered and may be referenced throughout the document.
\end{theorem}

\begin{lemma}
    Supporting results that aid in proving larger theorems are presented as lemmas.
    These are typically smaller, focused results.
\end{lemma}

\begin{corollary}
    Direct consequences of theorems or lemmas are presented as corollaries, representing results that follow immediately from previously established facts.
\end{corollary}

\begin{proposition}
    Mathematical statements of intermediate importance, less significant than theorems but more substantial than simple observations, are presented as propositions.
\end{proposition}

\begin{example}
    Concrete illustrations and applications of definitions, theorems, and concepts are provided in example environments to demonstrate practical usage.
\end{example}

\begin{axiom}
    Fundamental assumptions or rules that are accepted without proof are presented as axioms, forming the foundation for logical reasoning.
\end{axiom}

\begin{remark}
    Additional commentary, observations, or clarifications that supplement the main content are provided in remark environments.
\end{remark}

\begin{note}
    Important observations, warnings, or supplementary information that readers should particularly notice are highlighted in note environments.
\end{note}

\subsection*{Specialized Content Boxes}\label{subsec:specialized-content-boxes}

Three types of specialized content boxes are used for domain-specific material:

\begin{deduktcode}
    Code examples, syntax demonstrations, and programming constructs specific to the Dedukt system are presented in these highlighted code boxes. The syntax highlighting helps distinguish between different language elements.
\end{deduktcode}

\begin{inferenceBox}
    Formal inference rules, logical derivations, and reasoning patterns are presented in inference rule boxes.
    These typically show the structure of logical arguments and deductive steps.
\end{inferenceBox}

\begin{critique}
    Critical analysis, limitations, potential issues, or alternative viewpoints are presented in critique boxes.
    These provide balanced perspective and highlight important considerations.
\end{critique}


