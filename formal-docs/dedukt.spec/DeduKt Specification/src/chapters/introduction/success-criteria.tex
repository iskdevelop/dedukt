\section{Success Criteria}\label{sec:success-criteria}

The success of DeduKt will be measured through a combination of quantitative and qualitative metrics, which will assess its adoption, performance, quality, mathematical correctness, user experience, and community engagement.
These criteria provide a comprehensive framework for evaluating DeduKt’s progress and impact over time, helping to ensure that the project remains aligned with its mission while fostering growth and improvement.

\subsection{Quantitative Metrics}\label{subsec:quantitative-metrics}

\subsubsection{Adoption Metrics}

\begin{itemize}
    \item \textbf{Academic Adoption}: One of the key indicators of success is the extent to which DeduKt is adopted by academic institutions.
    By Year 5, DeduKt aims to be used in over 100 universities across the world, primarily in mathematics and related courses.
    This widespread use will validate DeduKt’s educational value and cement its role in academic curricula.

    \item \textbf{Research Usage}: By Year 5, DeduKt should have contributed to at least 500 published research papers, either as a tool for symbolic computation or for aiding in theoretical development.
    This will demonstrate the software’s acceptance and application in high-impact academic research.

    \item \textbf{Developer Community}: A strong developer community is critical for the growth of DeduKt.
    The project aims to attract over 1,000 active contributors by Year 4, ensuring continued development and expansion of its capabilities.
    A vibrant community will also enhance the software’s extensibility and integration with other scientific tools.

    \item \textbf{Download Statistics}: DeduKt will track its adoption among users through download metrics, with a goal of reaching 10,000+ monthly active users by Year 3.
    This will serve as a proxy for the software’s appeal and utility in the scientific and research communities.
\end{itemize}

\subsubsection{Performance Metrics}

\begin{itemize}
    \item \textbf{Computational Speed}: DeduKt aims to perform at a level comparable to the leading systems in symbolic computation, with a performance benchmark within 10\% of top systems for standard computational tests.
    This will ensure that DeduKt remains competitive in terms of processing speed for complex mathematical operations.

    \item \textbf{Memory Efficiency}: Memory usage will be optimized to be at least 25\% more efficient than interpreted alternatives, allowing DeduKt to handle large-scale computations with greater resource efficiency.
    This will be a key factor for researchers dealing with data-heavy problems.

    \item \textbf{Compilation Time}: The system will focus on ensuring quick compilation times for mathematical code.
    A target compilation time of under 5 seconds for typical projects will ensure that users experience minimal delays in their workflows.

    \item \textbf{Startup Time}: For interactive use, DeduKt will aim for a system startup time under 2 seconds, allowing users to immediately begin their computations without long waiting times.
\end{itemize}

\subsubsection{Quality Metrics}

\begin{itemize}
    \item \textbf{Bug Density}: To maintain the integrity of the software, DeduKt will strive for a bug density of fewer than 1 critical bug per 10,000 lines of mathematical code.
    This will ensure that the software remains stable and reliable for its users.

    \item \textbf{Test Coverage}: Comprehensive testing is crucial for ensuring the correctness of DeduKt's algorithms.
    The target is to achieve at least 95\% code coverage, including extensive testing of mathematical properties to ensure the system operates as intended across all use cases.

    \item \textbf{Documentation Coverage}: The project will aim for 100\% coverage of its API documentation, with clear and concise examples to guide users through the capabilities of DeduKt.
    Well-maintained documentation will ensure that users can easily learn and apply the software.

    \item \textbf{User Satisfaction}: User feedback will be collected annually, with a target user satisfaction rating of at least 90%.
    High satisfaction ratings will be indicative of the software’s effectiveness and the quality of its user experience.
\end{itemize}

\subsection{Qualitative Metrics}\label{subsec:qualitative-metrics}

\subsubsection{Mathematical Correctness}

\begin{itemize}
    \item \textbf{Formal Verification}: Core mathematical algorithms in DeduKt will undergo formal verification to ensure their correctness.
    This process will provide rigorous proof of the system’s reliability in performing symbolic computation.

    \item \textbf{Peer Review Validation}: Mathematical implementations will be subject to peer review, ensuring that DeduKt's computations are in line with established standards in the mathematical community.
    Cross-validation with well-known software like Mathematica or MATLAB will further validate DeduKt’s output.

    \item \textbf{Cross-Validation with Established Systems}: Regular comparisons will be made between DeduKt and established symbolic computation systems to verify that results remain consistent and mathematically valid.
    This will help maintain DeduKt’s reputation as a reliable computational tool.

    \item \textbf{Zero Tolerance for Mathematical Incorrectness}: A strict policy of zero tolerance for mathematical incorrectness will be maintained in all released versions of DeduKt.
    This will ensure that the software remains dependable for researchers and educators.
\end{itemize}

\subsubsection{User Experience Excellence}

\begin{itemize}
    \item \textbf{Intuitive Interfaces}: DeduKt will focus on providing intuitive interfaces that require minimal learning curves for domain experts.
    Whether using the command line or a basic GUI, the user experience should be seamless and easy to navigate for mathematicians and researchers.

    \item \textbf{Seamless Workflow Integration}: The software will be designed to integrate easily into existing mathematical research workflows, making it a natural addition to any researcher’s toolkit.
    This integration will be a key factor in DeduKt’s success in both academic and industry settings.

    \item \textbf{Positive Feedback from Mathematicians}: A key goal is to receive positive feedback from mathematicians transitioning from proprietary systems (e.g., Mathematica, MATLAB).
    Their endorsement will be a powerful indicator that DeduKt is providing value to the professional mathematical community.

    \item \textbf{Best-in-Class Development Experience}: DeduKt aims to be recognized as providing the best development experience for symbolic computation software.
    This includes user-friendly scripting environments, comprehensive error checking, and robust libraries for mathematical exploration.
\end{itemize}

\subsubsection{Community Health}

\begin{itemize}
    \item \textbf{Diverse, Inclusive Community}: DeduKt will foster a diverse and inclusive community that reflects the global mathematical community.
    The aim is to build a welcoming environment for contributors from all backgrounds, ensuring that the project thrives on collective input.

    \item \textbf{Active Mentorship Programs}: DeduKt will implement active mentorship programs to help new contributors find their footing.
    Experienced developers and mathematicians will provide guidance to ensure that new contributors are integrated into the project smoothly.

    \item \textbf{Transparent Governance}: The governance structure of DeduKt will be transparent, with clear processes for making major decisions.
    Community input will be solicited for key decisions to ensure that the project aligns with the needs of its users.

    \item \textbf{Sustainable Development Model}: A sustainable development model will ensure the long-term viability of DeduKt.
    This includes managing funding, community contributions, and development timelines to ensure that the software can continue evolving in the future.
\end{itemize}
