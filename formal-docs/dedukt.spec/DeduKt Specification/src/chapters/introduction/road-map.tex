\section{Roadmap Summary}\label{sec:roadmap-summary}
Production plan of DeduKt is based on its achievements not on time.
Below we explore phases and their expected outcome for DeduKt.

\subsection{Phase 0: Syntax and Interpreter}\label{subsec:phase-0:-syntax-and-interpreter}
Phase 0 marks the foundational stage in the development of DeduKt, concentrating on the core of the system: the syntax and interpreter.
This phase is essential as it forms the basis for how users will interact with the system and how the system will process and understand expressions.
The primary goal is to establish a robust, clear, and flexible foundation for all future developments.

\subsubsection{Key Objectives of Phase 0}
During Phase 0, the following critical activities will be carried out:
\begin{enumerate}
    \item \textbf{User-Experience Research on Readable Mathematical Scripts:}
    \begin{itemize}
        \item Conduct in-depth research into existing mathematical scripting languages (such as MATLAB, Mathematica, and SymPy) to evaluate their strengths and weaknesses in terms of usability, readability, and flexibility.
        \item Gather feedback from potential end-users—mathematicians, scientists, and engineers—on what makes a scripting language easy to read and understand in the context of complex mathematical formulations.
        \item Investigate how best to represent mathematical structures in code that is both syntactically correct and user-friendly.
        \item Explore the possibility of using natural language processing to enable easier interaction with the language in later phases.
    \end{itemize}

    \item \textbf{Development of DeduKt Grammar and Syntax:}
    \begin{itemize}
        \item Establish a formal grammar for DeduKt, defining the syntactic rules that will govern the language's structure.
        \item Decide on the core language constructs (e.g., expressions, functions, variables, data types) and how they will be represented within the system.
        \item Address edge cases and ambiguities in mathematical notation, ensuring consistency across the language.
        \item Provide extensibility in the syntax to accommodate future mathematical structures and computational needs.
    \end{itemize}

    \item \textbf{Development of Interpreter, Parser, and Lexer:}
    \begin{itemize}
        \item Build a \textbf{Lexer} to tokenize input scripts, breaking the raw code into manageable components such as numbers, variables, operators, and symbols.
        \item Implement a \textbf{Parser} that will transform tokenized input into an abstract syntax tree (AST), capturing the structure and logic of the mathematical expressions.
        \item Develop the \textbf{Interpreter} to execute mathematical operations and evaluations based on the parsed expressions.
        \item Ensure that the interpreter supports basic arithmetic, algebraic manipulation, and symbolic computation, forming the backbone of DeduKt's functionality.
    \end{itemize}
\end{enumerate}

\subsubsection{Expected Outcomes at the End of Phase 0}
By the end of Phase 0, we expect to have achieved the following outcomes:
\begin{enumerate}
    \item \textbf{Simple Arithmetic with Variables:}
    \begin{itemize}
        \item Users will be able to perform basic arithmetic operations (addition, subtraction, multiplication, division) on variables and constants.
        \item The system will support variables and the ability to manipulate symbolic expressions, forming the groundwork for more complex mathematical operations in later phases.
    \end{itemize}

    \item \textbf{Benchmark on Interpreter and Parser Outcomes:}
    \begin{itemize}
        \item The performance of the interpreter and parser will be benchmarked to assess their efficiency and correctness.
This includes testing the system with various expressions to ensure they are parsed and evaluated as expected.
        \item Initial tests will focus on parsing simple expressions, and later tests will expand to more complex symbolic operations as new features are added.
    \end{itemize}

    \item \textbf{Concrete Syntax Definition and Documentation:}
    \begin{itemize}
        \item A formal, detailed syntax definition will be established and documented, offering clear guidelines for how users can write DeduKt scripts.
        \item This documentation will serve as both a reference for developers and a user manual for future iterations of DeduKt.
        \item Examples of basic expressions and their corresponding syntactic forms will be included to ensure clarity.
    \end{itemize}
\end{enumerate}

\subsubsection{Deliverables of Phase 0}
The deliverables of Phase 0 will include:
\begin{itemize}
    \item A functional version of the interpreter, capable of evaluating simple mathematical expressions and supporting basic algebraic operations.
    \item A formal grammar and syntax specification for DeduKt, including rules for expression evaluation, variable declarations, and basic function definitions.
    \item A set of unit tests for the parser and interpreter to ensure correctness and performance.
    \item Initial user documentation covering the syntax, basic examples, and instructions for running simple computations.
\end{itemize}

\hline
\textbf{Other Phases would be added as we move towards them.}