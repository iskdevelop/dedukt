%! Author = kid-a
%! Date = 8/18/25

% Essential document setup
\documentclass[11pt,oneside,a4paper]{book}
\usepackage[utf8]{inputenc}
\usepackage[T1]{fontenc}
\usepackage[english]{babel}

% Page layout and formatting
\usepackage[top=2cm, bottom=2cm,left=2cm,right=2cm]{geometry}
\usepackage{fancyhdr}
\usepackage{titlesec}
\usepackage{tocloft}
\usepackage{setspace}
\usepackage{parskip}

% Mathematical typesetting
\usepackage{amsmath}
\usepackage{amssymb}
\usepackage{amsthm}
\usepackage{mathtools}
\usepackage{bm}           % Bold math symbols
\usepackage{dsfont}       % Double-struck fonts
\usepackage{stmaryrd}     % Additional math symbols

% Code and algorithms
\usepackage{listings}
\usepackage{minted}       % Superior syntax highlighting (requires pygments)
\usepackage{algorithm}
\usepackage{algpseudocode}
\usepackage{algorithmicx}
\usepackage{verbatim}
\usepackage{fancyvrb}     % Enhanced verbatim environments

% Graphics and figures
\usepackage{graphicx}
\usepackage{tikz}
\usepackage{pgfplots}
\usepackage{subcaption}
\usepackage{wrapfig}
\usepackage{float}

% TikZ libraries for diagrams
\usetikzlibrary{arrows.meta}
\usetikzlibrary{positioning}
\usetikzlibrary{shapes.geometric}
\usetikzlibrary{calc}
\usetikzlibrary{decorations.pathreplacing}
\usetikzlibrary{automata}
\usetikzlibrary{graphs}
\usetikzlibrary{trees}

% Tables and data presentation
\usepackage{booktabs}
\usepackage{tabularx}
\usepackage{longtable}
\usepackage{multirow}
\usepackage{array}
\usepackage{colortbl}

% Colors and styling
\usepackage{xcolor}
\usepackage{mdframed}
\usepackage{framed}
\usepackage{tcolorbox}
\tcbuselibrary{most}

% Cross-referencing and links
\usepackage{hyperref}
\usepackage{cleveref}
\usepackage{url}
\usepackage{bookmark}

% Bibliography and citations
\usepackage[style=alphabetic,backend=biber]{biblatex}
\usepackage{csquotes}


\usepackage{newfloat}
\usepackage{caption}

% Index and glossary
\usepackage{makeidx}
\usepackage[acronym,toc]{glossaries}

% Fonts (optional - choose one set)
% Modern Computer Modern
\usepackage{lmodern}
% OR for a more contemporary look:
% \usepackage{libertine}
% \usepackage[libertine]{newtxmath}

% Language-specific support for formal logic
\usepackage{proof}        % Natural deduction proofs
\usepackage{logicproof}   % Formal logic proofs
\usepackage{semantic}     % Semantic brackets

\definecolor{primary}{RGB}{112, 23, 255}      % Your primary color
\definecolor{primarytext}{RGB}{255, 255, 255} % White text on primary
\definecolor{surface}{RGB}{248, 248, 248}     % Light surface
\definecolor{surfacetext}{RGB}{22, 22, 22}    % Dark text on light
\definecolor{border}{RGB}{230, 230, 230}      % Subtle borders
\definecolor{accent}{RGB}{0, 0, 0}            % Black accent
\definecolor{muted}{RGB}{104, 104, 104}       % Muted text

% Custom theorem environments with IBM Design principles
\theoremstyle{definition}
\newtheorem{definition}{Definition}[chapter]
\newtheorem{example}{Example}[chapter]
\newtheorem{axiom}{Axiom}[chapter]

\theoremstyle{plain}
\newtheorem{theorem}{Theorem}[chapter]
\newtheorem{proposition}{Proposition}[chapter]
\newtheorem{lemma}{Lemma}[chapter]
\newtheorem{corollary}{Corollary}[chapter]

\theoremstyle{remark}
\newtheorem{remark}{Remark}[chapter]
\newtheorem{note}{Note}[chapter]

% IBM Design inspired code colors
\definecolor{codegreen}{RGB}{22, 101, 52}     % IBM green
\definecolor{codegray}{RGB}{104, 104, 104}    % IBM gray
\definecolor{codepurple}{RGB}{112, 23, 255}   % Your primary
\definecolor{codebackground}{RGB}{248, 248, 248}  % Clean background

% Listings configuration with IBM Design aesthetics
\lstdefinestyle{dedukt}{
    backgroundcolor=\color{codebackground},
    commentstyle=\color{codegreen}\itshape,
    keywordstyle=\color{primary}\bfseries,
    numberstyle=\tiny\color{codegray},
    stringstyle=\color{codepurple},
    basicstyle=\ttfamily\footnotesize\color{surfacetext},
    breakatwhitespace=false,
    breaklines=true,
    captionpos=b,
    keepspaces=true,
    numbers=left,
    numbersep=8pt,
    showspaces=false,
    showstringspaces=false,
    showtabs=false,
    tabsize=2,
    frame=none,                    % No rounded frames
    rulecolor=\color{border},
    xleftmargin=12pt,
    xrightmargin=12pt,
    aboveskip=12pt,
    belowskip=12pt
}

\lstset{style=dedukt}

% IBM Design inspired tcolorbox styles - minimal, sharp, high contrast
\newtcolorbox{deduktcode}{
    colback=codebackground,
    colframe=primary,
    coltitle=primarytext,
    colbacktitle=primary,
    boxrule=1pt,
    sharp corners,              % No roundness
    boxsep=8pt,
    left=8pt,
    right=8pt,
    top=8pt,
    bottom=8pt,
    title={\bfseries Dedukt Code},
    fonttitle=\sffamily\bfseries,
    enhanced,
    attach boxed title to top left={
        xshift=0pt,
        yshift=-\tcboxedtitleheight/2
    },
    boxed title style={
        sharp corners,
        boxrule=1pt,
        colframe=primary,
        colback=primary
    }
}

\newtcolorbox{inferenceBox}{
    colback=surface,
    colframe=surface,
    coltitle=surfacetext,
    colbacktitle=white,
    boxrule=1pt,
    sharp corners,              % No roundness
    boxsep=8pt,
    left=8pt,
    right=8pt,
    top=8pt,
    bottom=8pt,
    title={\bfseries Inference Rule},
    enhanced,
    attach boxed title to top left={
        xshift=0pt,
        yshift=-\tcboxedtitleheight/2
    },
    boxed title style={
        sharp corners,
        boxrule=1pt,
        colframe=surface,
        colback=surface
    }
}

\newtcolorbox{critique}{
    colback=surface,
    colframe=accent,
    coltitle=white,
    colbacktitle=accent,
    boxrule=1pt,
    sharp corners,              % No roundness
    boxsep=8pt,
    left=8pt,
    right=8pt,
    top=8pt,
    bottom=8pt,
    title={\bfseries Critical Analysis},
    fonttitle=\sffamily\bfseries,
    enhanced,
    attach boxed title to top left={
        xshift=0pt,
        yshift=-\tcboxedtitleheight/2
    },
    boxed title style={
        sharp corners,
        boxrule=1pt,
        colframe=accent,
        colback=accent
    }
}

% IBM Design inspired hyperref setup - clean, minimal
\hypersetup{
    colorlinks=true,
    linkcolor=primary,          % Use your primary color
    filecolor=primary,
    urlcolor=primary,
    citecolor=accent,
    pdftitle={Dedukt: A System for Symbolic Reasoning},
    pdfauthor={Your Name},
    pdfsubject={Symbolic Reasoning and Computer Algebra},
    pdfkeywords={dedukt, symbolic reasoning, formal logic, computer algebra},
    pdfborder={0 0 0},         % No border around links
    bookmarksnumbered=true,
    bookmarksopen=false
}



% Clean table styling
\arrayrulecolor{border}
\setlength{\arrayrulewidth}{0.5pt}

% Minimal footnote styling
\renewcommand{\footnoterule}{%
    \kern -3pt
    \textcolor{border}{\hrule width 2in height 0.5pt}
    \kern 2.6pt
}
\makeglossary
\makeindex
% Fix pgfplots compatibility warning
\pgfplotsset{compat=1.18}

% Bibliography setup (move from document body to preamble)
% If using biblatex (recommended):
\addbibresource{main.bib}  % Note: .bib extension required for biblatex
% Document
\begin{document}
    \begin{titlepage}
        \centering
        \vspace*{2cm}

        {\Huge\bfseries DeduKt \par}
        \vspace{1cm}
        {\Large Language of Reasoning \par}
        \vspace{2cm}

        {\Large Version 0.0.1 (Draft)\par}
        \vspace{0.5cm}
        {\large \today \par}

        \vfill

        % Optional: Author or organization
        {\Large Independent Society of Knowledge (ISK) \par}

    \end{titlepage}

    \tableofcontents


    \section*{Authors / Contributors / Affiliations}
\begin{itemize}
    \item Amir H. Ebarhimnezhad \\
    Founder, Independent Society of Knowledge (ISK)
\end{itemize}

    \newpage
    \section*{Version}
\begin{tabular}{ll}
    Version: & 0.0.1 \\
    Date: & 18 August 2025 \\
\end{tabular}

    \newpage
    \section*{License \& Copyright}

\noindent
Copyright (c) 2025 Amir H. Ebarhimnezhad.
All rights reserved.

\noindent
This document and its contents are licensed under the Apache License, Version 2.0.
You may obtain a copy of the License at: \url{http://www.apache.org/licenses/LICENSE-2.0}

\noindent
Unless required by applicable law or agreed to in writing, software and documentation
distributed under the License is distributed on an \emph{AS IS} BASIS, WITHOUT WARRANTIES
OR CONDITIONS OF ANY KIND, either express or implied.
See the License for the specific language
governing permissions and limitations under the License.

    \newpage
    \section*{Abstract}

DeduKt is an open-source, community-driven system designed for computer algebra, simulation, and data manipulation.
It provides a dynamic and expressive language for modeling and defining mathematics, leveraging Kotlin, a modern, statically typed, general-purpose language, as the main development language.
Additionally, DeduKt includes a library for symbolic computation within Kotlin, enabling both native integration and advanced mathematical reasoning.

    \newpage
    \section*{Preface}\label{sec:preface}
This document serves as the official guide to DeduKt, a system of symbolic reasoning designed to push the boundaries of computational mathematics, algebra, and symbolic manipulation.
DeduKt is built with a core philosophy rooted in precision, efficiency, and modularity, and it aims to provide an alternative to existing symbolic computation systems like Mathematica, MATLAB, and SageMath, offering both flexibility and performance.


The goal of this documentation is twofold: to provide an in-depth explanation of the design and functionality of DeduKt, and to serve as a comprehensive resource for users, developers, and researchers seeking to understand the system at both the conceptual and technical levels.


In the following chapters, we will explore the philosophy that drives DeduKt, diving into the underlying principles that informed its creation.
We will then proceed to examine the system design, detailing the components that make up the framework, and providing insight into how symbolic computation is handled and optimized.
Special emphasis is placed on language design, including the lexical grammar, syntax, and evaluation model of the DeduKt language.


This document also covers a wide array of practical aspects.
The compiler design is thoroughly explained, shedding light on how symbolic expressions are parsed, compiled, and executed. Additionally, we provide a usage guide with detailed instructions on installation, basic usage, and more advanced topics, helping users maximize the system’s potential.


Finally, we offer a critical evaluation of existing symbolic computation systems, identifying their strengths and weaknesses, and demonstrating how DeduKt builds upon, and in some cases, improves upon these legacy tools.
The critiques of Mathematica, MATLAB, and SageMath provide a broader context within which DeduKt’s design decisions were made.


The appendices include a glossary of terms, further reading, complete code listings, and licensing details, making this document a valuable reference for both novice users and experienced developers.


Whether you are here to explore DeduKt as a user, developer, or researcher, this documentation is intended to provide a complete understanding of the system, its features, and its potential.
By the end of this document, we hope to have conveyed not only how DeduKt works but also why it exists — as an answer to the limitations of traditional symbolic computation systems and a step toward more open, flexible, and powerful tools for mathematical reasoning.


    \newpage
    \section*{Acknowledgements}

The author would like to thank the following individuals for their guidance, support, and contributions to the DeduKt project:

\begin{itemize}
    \item Narges Haji Rasouliha, PhD, University of Tehran
    \item Ehsan Kiani, Independent Society of Knowledge
    \item Danial Yahyazadeh, Independent Society of Knowledge
\end{itemize}

    \newpage
    \section*{Conventions and Notation}\label{sec:conventions-and-notation}

\subsection*{Hierarchical Organization}\label{subsec:hierarchical-organization}
The document is structured using a clear hierarchical system:
\begin{itemize}
    \item \textbf{Chapters} represent major topics and are numbered sequentially
    \item \textbf{Sections} divide chapters into primary subtopics
    \item \textbf{Subsections} provide further subdivision of complex topics
    \item \textbf{Subsubsections} offer detailed breakdowns when necessary
\end{itemize}

All theorems, definitions, examples, and other mathematical constructs are numbered within their respective chapters using the format \texttt{Type Chapter.Number} (e.g., Definition 2.1, Theorem 3.5).

\subsection*{Special Environments}\label{subsec:special-environments}

The document employs several specialized environments to categorize different types of content:

\begin{definition}
    Formal definitions of terms, concepts, and mathematical objects are presented in definition boxes like this one.
    Definitions establish precise meanings for terminology used throughout the document.
\end{definition}

\begin{theorem}
    Mathematical statements that have been proven are presented as theorems.
    Each theorem is numbered and may be referenced throughout the document.
\end{theorem}

\begin{lemma}
    Supporting results that aid in proving larger theorems are presented as lemmas.
    These are typically smaller, focused results.
\end{lemma}

\begin{corollary}
    Direct consequences of theorems or lemmas are presented as corollaries, representing results that follow immediately from previously established facts.
\end{corollary}

\begin{proposition}
    Mathematical statements of intermediate importance, less significant than theorems but more substantial than simple observations, are presented as propositions.
\end{proposition}

\begin{example}
    Concrete illustrations and applications of definitions, theorems, and concepts are provided in example environments to demonstrate practical usage.
\end{example}

\begin{axiom}
    Fundamental assumptions or rules that are accepted without proof are presented as axioms, forming the foundation for logical reasoning.
\end{axiom}

\begin{remark}
    Additional commentary, observations, or clarifications that supplement the main content are provided in remark environments.
\end{remark}

\begin{note}
    Important observations, warnings, or supplementary information that readers should particularly notice are highlighted in note environments.
\end{note}

\subsection*{Specialized Content Boxes}\label{subsec:specialized-content-boxes}

Three types of specialized content boxes are used for domain-specific material:

\begin{deduktcode}
    Code examples, syntax demonstrations, and programming constructs specific to the Dedukt system are presented in these highlighted code boxes. The syntax highlighting helps distinguish between different language elements.
\end{deduktcode}

\begin{inferenceBox}
    Formal inference rules, logical derivations, and reasoning patterns are presented in inference rule boxes.
    These typically show the structure of logical arguments and deductive steps.
\end{inferenceBox}

\begin{critique}
    Critical analysis, limitations, potential issues, or alternative viewpoints are presented in critique boxes.
    These provide balanced perspective and highlight important considerations.
\end{critique}

\subsection*{Mathematical Notation}\label{subsec:mathematical-notation}
Mathematical expressions follow standard conventions:
\begin{itemize}
    \item Variables are typically represented by lowercase italic letters: $x, y, z, a, b, c$
    \item Constants are represented by specific symbols or uppercase letters when contextually clear
    \item Functions are denoted by lowercase letters followed by parentheses: $f(x), g(y), h(z)$
    \item Sets are represented by uppercase italic letters: $A, B, C, S, T$
    \item Mathematical operators follow standard notation: $+, -, \times, \div, =, \neq, <, >, \leq, \geq$
\end{itemize}

Standard set theory symbols are used consistently:

\begin{align}
    \emptyset &\quad \text{empty set} \\
    \in &\quad \text{element of} \\
    \notin &\quad \text{not an element of} \\
    \subset &\quad \text{proper subset} \\
    \subseteq &\quad \text{subset or equal} \\
    \cup &\quad \text{union} \\
    \cap &\quad \text{intersection} \\
    \setminus &\quad \text{set difference} \\
    |A| &\quad \text{cardinality of set } A
\end{align}

Logical operators and quantifiers follow standard conventions:
\begin{align}
    \neg &\quad \text{negation (not)} \\
    \land &\quad \text{conjunction (and)} \\
    \lor &\quad \text{disjunction (or)} \\
    \rightarrow &\quad \text{implication (if-then)} \\
    \leftrightarrow &\quad \text{biconditional (if and only if)} \\
    \forall &\quad \text{universal quantifier (for all)} \\
    \exists &\quad \text{existential quantifier (there exists)} \\
    \exists! &\quad \text{unique existence (there exists exactly one)}
\end{align}

When discussing the Dedukt system specifically, certain conventions apply:
\begin{itemize}
    \item Dedukt expressions are typeset in monospace font within the text: \texttt{expr}
    \item Multi-line Dedukt code is presented in dedicated code environments
    \item Dedukt keywords and operators retain their system-specific syntax
    \item Type annotations and signatures follow Dedukt's formal specification
\end{itemize}

\subsection*{Typographical Conventions}\label{subsec:typographical-conventions}
\begin{itemize}
    \item \textbf{Bold text} indicates important terms, key concepts, or emphasis
    \item \textit{Italic text} is used for mathematical variables, technical terms being defined, or subtle emphasis
    \item \texttt{Monospace text} represents code, system commands, filenames, or literal computer input/output
    \item \textsf{Sans-serif text} is used sparingly for interface elements or modern technical terms
\end{itemize}

\begin{itemize}
    \item Internal references use descriptive text with hyperlinks to relevant sections, theorems, or definitions
    \item External references follow standard academic citation format
    \item Theorem and definition references include both number and descriptive context when helpful
    \item Code listings and figures are numbered and may be referenced by number
\end{itemize}

\begin{itemize}
    \item Inline code uses monospace formatting: \texttt{variable\_name}
    \item Code blocks preserve original formatting and include syntax highlighting where applicable
    \item Mathematical expressions within code environments maintain monospace formatting
    \item Output examples are clearly distinguished from input code
\end{itemize}

\subsection*{Visual Design Principles}\label{subsec:visual-design-principles}

The document's visual design emphasizes clarity, consistency, and accessibility:

\begin{itemize}
    \item Primary accent color is used consistently for headings, borders, and highlights
    \item High contrast ensures readability in both print and digital formats
    \item Color coding in syntax highlighting follows conventional programming standards
    \item Color is never the sole means of conveying information
\end{itemize}

\begin{itemize}
    \item Consistent vertical spacing between elements maintains visual rhythm
    \item Code blocks are clearly separated from surrounding text
    \item Mathematical expressions are properly spaced for readability
    \item White space is used effectively to group related content and separate distinct concepts
\end{itemize}

\begin{itemize}
    \item Sharp corners and clean lines reflect modern, minimal design principles
    \item Border weights are consistent across similar element types
    \item Background colors provide sufficient contrast without overwhelming the content
    \item Title bars clearly identify the purpose of specialized environments
\end{itemize}



    \chapter{Introduction}\label{ch:introduction}
    \section{Overview}\label{sec:overview}
In mathematics and the natural sciences, computers serve not only as powerful tools for simulations and numerical computation but also as essential systems for symbolic manipulation of mathematical expressions.
Despite the many existing Computer Algebra Systems (CAS) and simulation software available today, they tend to follow different philosophies and payment plans that reduces their capabilities and community.
DeduKt is created to address and solve these issues.

DeduKt is an open-source, community-driven system designed for computer algebra, simulation, and data manipulation.
It provides a dynamic and expressive language for modeling and defining mathematics, leveraging Kotlin, a modern, statically typed, general-purpose language, as the main language of development (Unless others specified) and having a library for symbolic computation inside of it.
Through its unique approach and the connection between DeduKt and Independent Society of Knowledge's Kompute library, DeduKt aims to become a unified software for reasoning, modeling and quantitative researches.

While numerous powerful software systems like Wolfram Mathematica, COMSOL, and SageMath dominate the computational landscape, they often suffer from limitations such as cost, complexity, and closed-source nature.
DeduKt stands as an open alternative, offering both the flexibility and extensibility needed for modern computational workflows.
It aims to break free from the constraints of proprietary software, ensuring that users have complete control over their computational environment while benefiting from the power and robustness of a community-driven open-source platform.

At its core, DeduKt is designed to be a hybrid tool that can be used both as a standalone software package and as a Kotlin library, enabling seamless integration into existing Kotlin-based workflows.
With Kompute, DeduKt's numerical computing counterpart, the system supports a wide range of applications—from symbolic computation to advanced simulation tasks—while maintaining a focus on performance, usability, and scalability.

As a community-driven project, DeduKt's future is shaped by the contributions and feedback of its users.
Whether you're a researcher, developer, or educator, DeduKt offers a flexible, extensible, and powerful tool set for advancing computational mathematics and science.

In the following section we would give an introduction to DeduKt as a project, we would explain driving philosophies in design, architecture, user-experience and development.
Later chapters would cover these topics in great detail and therefore, this introduction serves as an overview to the whole document.

    \section{Mission Statement}\label{sec:mission-statement}
In the landscape of scientific computation, the types of computation are traditionally divided into two broad categories: numerical and symbolic.
While numerical methods have long been at the forefront of computational science, symbolic computation remains essential for the theoretical underpinnings of many scientific domains.
DeduKt is designed to provide a comprehensive solution for symbolic computation, while maintaining a seamless integration with numerical methods.
This duality is crucial as many scientific workflows rely on the interplay between abstract, symbolic reasoning and empirical, numerical analysis.

The fundamental premise of DeduKt is to empower users to compute and manipulate mathematical expressions within their correct contextual framework.
Unlike conventional tools that focus solely on the evaluation of expressions, DeduKt emphasizes the need for intelligent reasoning throughout the computational process.
This means not only evaluating equations, but also offering insights, suggestions, and paths for exploration that can guide users through their computations.
By understanding the broader context of the problem, DeduKt allows users to reason, experiment, and iterate more effectively.

Moreover, modern scientific workflows are increasingly data-driven.
While theoretical models provide critical insights, the need to analyze and interpret data from experiments is equally important.
DeduKt aims to be a bridge between theory and experiment, enabling users to seamlessly transition from mathematical derivations to real-world data analysis.
The ability to integrate symbolic computation with experimental data handling ensures that DeduKt remains versatile and functional in a wide range of research scenarios.

The goal of DeduKt is to provide the following key capabilities:

\begin{itemize}
    \item An extensible and modular framework for mathematical computation and symbolic manipulation, designed to easily accommodate new mathematical structures, algorithms, and evaluation methods as they evolve in the field.
    \item A powerful toolkit for data manipulation, processing, and visualization, enabling users to analyze experimental data alongside their symbolic models.
    This toolkit will support a range of formats and provide intuitive interfaces for data exploration.
    \item Intelligent reasoning systems that not only ensure the correctness of computations but also assist in the development of rigorous mathematical proofs, theorem generation, and symbolic simplifications.
    This will empower users to conduct research at the forefront of mathematical exploration.
    \item A commitment to being a free, open-source software project, fostering a vibrant community of contributors, educators, and researchers.
    This ensures that DeduKt remains accessible, adaptable, and continuously improved by the collective efforts of its user base.
    \item Complementary integration with ISK's Numerical Foundation Library (Kompute), providing a unified computational environment for both numerical and symbolic tasks.
    This synergy will allow users to approach problems from both a theoretical and experimental perspective without needing to switch tools.
    \item A minimalistic core design focused on efficiency, performance, and flexibility, while offering users the ability to define their own symbolic computation methods.
    The system will prioritize a clean, user-friendly interface, even as it supports complex and customizable use cases.
    \item High performance and scalability, ensuring that DeduKt can handle large-scale computations and complex symbolic manipulations without sacrificing speed or responsiveness.
    This includes optimizations for both single-user applications and collaborative, distributed environments.
\end{itemize}

    \section{Non-Goals}\label{sec:non-goals}
While DeduKt is designed to be a powerful tool for symbolic computation and mathematical reasoning, there are several areas that are intentionally outside the current scope of the project.
These non-goals represent areas where DeduKt does not aim to provide functionality or where certain capabilities are deliberately excluded to maintain focus on its core mission.
It is important to note that the scope of DeduKt is subject to change as the project evolves, and if the mission statement shifts, these non-goals may be revisited or expanded.

The following are key aspects that are not within the current scope of the DeduKt project:

\begin{itemize}
    \item \textbf{Numerical Simulation and High-Performance Computing (HPC)}: While DeduKt integrates symbolic computation with numerical data analysis, it does not aim to provide comprehensive numerical simulation or high-performance computing features.
    For advanced numerical tasks like large-scale simulations or parallel processing of mathematical models, users are encouraged to rely on specialized numerical software like Kompute or other HPC tools.

    \item \textbf{Graphical User Interface (GUI)}: DeduKt is primarily focused on providing a powerful computational engine for symbolic reasoning.
    While the system may include some minimal user interfaces for basic interaction, DeduKt does not currently prioritize the development of a full-featured GUI.
    The emphasis remains on the core computational engine and command-line or script-based interfaces.

    \item \textbf{Advanced Machine Learning or AI Integration}: DeduKt is not intended to serve as a platform for advanced machine learning or artificial intelligence applications.
    While the software may allow users to integrate or manipulate data produced by machine learning algorithms, it does not include native support for training models, neural networks, or deep learning techniques.
    For AI-driven tasks, users should turn to specialized frameworks like TensorFlow or PyTorch.

    \item \textbf{General-Purpose Programming Language Features}: DeduKt is not designed to be a general-purpose programming language.
    Although it supports some user-defined symbolic computation and custom operations, it does not intend to compete with full-fledged programming languages like Python, Go, or Kotlin for broad software development tasks.
    DeduKt’s focus is on symbolic mathematics and mathematical reasoning, not on general software engineering.

    \item \textbf{Real-Time Collaboration Features}: While DeduKt may be used in a collaborative research setting, it does not currently offer features for real-time collaborative editing or communication.
    Users looking for collaborative environments should rely on other tools designed for shared workflows, such as Jupyter notebooks, or consider using DeduKt alongside version control systems like Git.

    \item \textbf{Specialized Visualization Tools for Complex Data Sets}: While DeduKt includes basic data manipulation and visualization tools for mathematical expressions, it does not aim to replace specialized data visualization libraries or platforms.
    Complex visualizations such as interactive 3D plots, real-time data streams, or extensive dashboard frameworks are beyond the scope of the project.
    For advanced visualization, users are encouraged to integrate DeduKt with tools like Matplotlib, Plotly, or specialized scientific visualization software.

    \item \textbf{Commercial or Enterprise Support}: DeduKt is an open-source software project, and as such, it does not provide official commercial or enterprise-grade support.
    While the community-driven approach fosters collaboration and assistance, users requiring enterprise-level guarantees, support contracts, or specific service-level agreements (SLAs) should seek professional services elsewhere.

    \item \textbf{Native Support for All Mathematical Fields}: Although DeduKt is designed to handle a wide range of symbolic mathematical tasks, it does not support every niche or specialized area of mathematics.
    Areas requiring highly specialized algorithms, such as advanced cryptography or domain-specific symbolic methods, may not be fully supported at the outset.
    However, the extensible nature of DeduKt allows users to implement their own methods for these areas.
\end{itemize}

    \section{Target Audience}\label{sec:target-audience}

DeduKt is designed with a specific set of users in mind: those who require symbolic computation and mathematical reasoning as part of their research or scientific work.
However, its capabilities and focus areas also determine the audiences it is not targeted towards.
Below is a breakdown of the primary users and those outside of the scope of DeduKt.

\subsection{Primary Target Audience}\label{subsec:primary-target-audience}

The primary target audience for DeduKt includes:

\begin{itemize}
    \item \textbf{Researchers and Academics in Mathematics and Physics}: DeduKt is ideally suited for researchers and students working in fields that rely on symbolic mathematics, such as algebra, calculus, differential equations, and other areas of pure and applied mathematics.
Its advanced symbolic reasoning capabilities and extensibility make it an excellent tool for conducting mathematical proofs, solving complex equations, and developing new mathematical theories.

    \item \textbf{Scientific Engineers and Data Analysts}: DeduKt bridges the gap between theoretical mathematics and real-world data analysis, making it a valuable tool for engineers and scientists who need to manipulate and reason with both symbolic and experimental data.
For example, researchers in computational physics, machine learning, or data science can use DeduKt to work with mathematical models while integrating experimental results into their workflows.

    \item \textbf{Open-Source and Community-Oriented Contributors}: DeduKt’s open-source nature and modular design make it an attractive platform for contributors who are interested in developing or enhancing symbolic computation libraries.
Researchers, software developers, and hobbyists who are passionate about advancing the state of open scientific software can contribute to DeduKt’s continued evolution.

    \item \textbf{Developers with an Interest in Symbolic Computation}: DeduKt is also suitable for software developers who are looking to integrate symbolic computation into their own applications.
The system’s extensibility and ability to define custom symbolic operations make it a useful resource for those building mathematical software or libraries.
\end{itemize}

\subsection{Secondary Target Audience}\label{subsec:secondary-target-audience}

The secondary target audience includes:

\begin{itemize}
    \item \textbf{Students in Computational Mathematics or Theoretical Physics}: DeduKt is a great tool for students learning computational techniques, symbolic manipulation, and mathematical modeling.
It can serve as both an educational tool and a research assistant for students in mathematics and physics programs.

    \item \textbf{Data Science and Machine Learning Enthusiasts}: While DeduKt is not designed as a full-fledged data science or machine learning platform, data scientists and ML practitioners who need symbolic computation for pre-processing, feature engineering, or model exploration will benefit from DeduKt’s capabilities to handle both theory and data in one platform.
\end{itemize}

\subsection{Not Targeted Audience}\label{subsec:not-targeted-audience}

DeduKt is not designed for the following audiences:

\begin{itemize}
    \item \textbf{General Software Developers}: While DeduKt allows for the extension of symbolic computation capabilities, it is not a general-purpose programming language like Kotlin, Go, or Python.
Developers seeking to build non-symbolic applications or software with no focus on mathematical reasoning or symbolic manipulation are not the primary users of DeduKt.

    \item \textbf{Users Seeking GUI-Based Software for Scientific Computing}: DeduKt is not focused on providing graphical interfaces for users.
While minimal interfaces may be provided for basic interaction, DeduKt’s core functionality is designed for script-based interaction and command-line execution.
Those who require full-fledged graphical user interfaces for interactive scientific computing will find other platforms, such as MATLAB or Mathematica, more suited to their needs.

    \item \textbf{Commercial Enterprises Requiring Dedicated Support}: DeduKt is an open-source software, and as such, it does not provide official commercial support or service-level agreements (SLAs).
Enterprises looking for a commercial product with dedicated support and guaranteed uptime should consider other options that cater specifically to enterprise needs.

    \item \textbf{Non-Mathematical Users or Casual Enthusiasts}: DeduKt is designed for advanced users dealing with mathematical reasoning and symbolic computation.
Casual users or individuals seeking software for basic tasks, such as word processing or general-purpose calculations, will not find DeduKt appropriate for their needs.

    \item \textbf{Advanced Machine Learning or AI Practitioners}: Although DeduKt may handle basic data manipulation tasks, it does not offer comprehensive tools for machine learning, neural networks, or advanced AI research.
Users interested in deep learning or complex machine learning pipelines will need to rely on specialized frameworks like TensorFlow or PyTorch.
\end{itemize}

    \section{Success Criteria}\label{sec:success-criteria}

The success of DeduKt will be measured through a combination of quantitative and qualitative metrics, which will assess its adoption, performance, quality, mathematical correctness, user experience, and community engagement.
These criteria provide a comprehensive framework for evaluating DeduKt’s progress and impact over time, helping to ensure that the project remains aligned with its mission while fostering growth and improvement.

\subsection{Quantitative Metrics}\label{subsec:quantitative-metrics}

\subsubsection{Adoption Metrics}

\begin{itemize}
    \item \textbf{Academic Adoption}: One of the key indicators of success is the extent to which DeduKt is adopted by academic institutions.
    By Year 5, DeduKt aims to be used in over 100 universities across the world, primarily in mathematics and related courses.
    This widespread use will validate DeduKt’s educational value and cement its role in academic curricula.

    \item \textbf{Research Usage}: By Year 5, DeduKt should have contributed to at least 500 published research papers, either as a tool for symbolic computation or for aiding in theoretical development.
    This will demonstrate the software’s acceptance and application in high-impact academic research.

    \item \textbf{Developer Community}: A strong developer community is critical for the growth of DeduKt.
    The project aims to attract over 1,000 active contributors by Year 4, ensuring continued development and expansion of its capabilities.
    A vibrant community will also enhance the software’s extensibility and integration with other scientific tools.

    \item \textbf{Download Statistics}: DeduKt will track its adoption among users through download metrics, with a goal of reaching 10,000+ monthly active users by Year 3.
    This will serve as a proxy for the software’s appeal and utility in the scientific and research communities.
\end{itemize}

\subsubsection{Performance Metrics}

\begin{itemize}
    \item \textbf{Computational Speed}: DeduKt aims to perform at a level comparable to the leading systems in symbolic computation, with a performance benchmark within 10\% of top systems for standard computational tests.
    This will ensure that DeduKt remains competitive in terms of processing speed for complex mathematical operations.

    \item \textbf{Memory Efficiency}: Memory usage will be optimized to be at least 25\% more efficient than interpreted alternatives, allowing DeduKt to handle large-scale computations with greater resource efficiency.
    This will be a key factor for researchers dealing with data-heavy problems.

    \item \textbf{Compilation Time}: The system will focus on ensuring quick compilation times for mathematical code.
    A target compilation time of under 5 seconds for typical projects will ensure that users experience minimal delays in their workflows.

    \item \textbf{Startup Time}: For interactive use, DeduKt will aim for a system startup time under 2 seconds, allowing users to immediately begin their computations without long waiting times.
\end{itemize}

\subsubsection{Quality Metrics}

\begin{itemize}
    \item \textbf{Bug Density}: To maintain the integrity of the software, DeduKt will strive for a bug density of fewer than 1 critical bug per 10,000 lines of mathematical code.
    This will ensure that the software remains stable and reliable for its users.

    \item \textbf{Test Coverage}: Comprehensive testing is crucial for ensuring the correctness of DeduKt's algorithms.
    The target is to achieve at least 95\% code coverage, including extensive testing of mathematical properties to ensure the system operates as intended across all use cases.

    \item \textbf{Documentation Coverage}: The project will aim for 100\% coverage of its API documentation, with clear and concise examples to guide users through the capabilities of DeduKt.
    Well-maintained documentation will ensure that users can easily learn and apply the software.

    \item \textbf{User Satisfaction}: User feedback will be collected annually, with a target user satisfaction rating of at least 90%.
    High satisfaction ratings will be indicative of the software’s effectiveness and the quality of its user experience.
\end{itemize}

\subsection{Qualitative Metrics}\label{subsec:qualitative-metrics}

\subsubsection{Mathematical Correctness}

\begin{itemize}
    \item \textbf{Formal Verification}: Core mathematical algorithms in DeduKt will undergo formal verification to ensure their correctness.
    This process will provide rigorous proof of the system’s reliability in performing symbolic computation.

    \item \textbf{Peer Review Validation}: Mathematical implementations will be subject to peer review, ensuring that DeduKt's computations are in line with established standards in the mathematical community.
    Cross-validation with well-known software like Mathematica or MATLAB will further validate DeduKt’s output.

    \item \textbf{Cross-Validation with Established Systems}: Regular comparisons will be made between DeduKt and established symbolic computation systems to verify that results remain consistent and mathematically valid.
    This will help maintain DeduKt’s reputation as a reliable computational tool.

    \item \textbf{Zero Tolerance for Mathematical Incorrectness}: A strict policy of zero tolerance for mathematical incorrectness will be maintained in all released versions of DeduKt.
    This will ensure that the software remains dependable for researchers and educators.
\end{itemize}

\subsubsection{User Experience Excellence}

\begin{itemize}
    \item \textbf{Intuitive Interfaces}: DeduKt will focus on providing intuitive interfaces that require minimal learning curves for domain experts.
    Whether using the command line or a basic GUI, the user experience should be seamless and easy to navigate for mathematicians and researchers.

    \item \textbf{Seamless Workflow Integration}: The software will be designed to integrate easily into existing mathematical research workflows, making it a natural addition to any researcher’s toolkit.
    This integration will be a key factor in DeduKt’s success in both academic and industry settings.

    \item \textbf{Positive Feedback from Mathematicians}: A key goal is to receive positive feedback from mathematicians transitioning from proprietary systems (e.g., Mathematica, MATLAB).
    Their endorsement will be a powerful indicator that DeduKt is providing value to the professional mathematical community.

    \item \textbf{Best-in-Class Development Experience}: DeduKt aims to be recognized as providing the best development experience for symbolic computation software.
    This includes user-friendly scripting environments, comprehensive error checking, and robust libraries for mathematical exploration.
\end{itemize}

\subsubsection{Community Health}

\begin{itemize}
    \item \textbf{Diverse, Inclusive Community}: DeduKt will foster a diverse and inclusive community that reflects the global mathematical community.
    The aim is to build a welcoming environment for contributors from all backgrounds, ensuring that the project thrives on collective input.

    \item \textbf{Active Mentorship Programs}: DeduKt will implement active mentorship programs to help new contributors find their footing.
    Experienced developers and mathematicians will provide guidance to ensure that new contributors are integrated into the project smoothly.

    \item \textbf{Transparent Governance}: The governance structure of DeduKt will be transparent, with clear processes for making major decisions.
    Community input will be solicited for key decisions to ensure that the project aligns with the needs of its users.

    \item \textbf{Sustainable Development Model}: A sustainable development model will ensure the long-term viability of DeduKt.
    This includes managing funding, community contributions, and development timelines to ensure that the software can continue evolving in the future.
\end{itemize}

    \section{Roadmap Summary}\label{sec:roadmap-summary}
Production plan of DeduKt is based on its achievements not on time.
Below we explore phases and their expected outcome for DeduKt.

\subsection{Phase 0: Syntax and Interpreter}\label{subsec:phase-0:-syntax-and-interpreter}
Phase 0 marks the foundational stage in the development of DeduKt, concentrating on the core of the system: the syntax and interpreter.
This phase is essential as it forms the basis for how users will interact with the system and how the system will process and understand expressions.
The primary goal is to establish a robust, clear, and flexible foundation for all future developments.

\subsubsection{Key Objectives of Phase 0}
During Phase 0, the following critical activities will be carried out:
\begin{enumerate}
    \item \textbf{User-Experience Research on Readable Mathematical Scripts:}
    \begin{itemize}
        \item Conduct in-depth research into existing mathematical scripting languages (such as MATLAB, Mathematica, and SymPy) to evaluate their strengths and weaknesses in terms of usability, readability, and flexibility.
        \item Gather feedback from potential end-users—mathematicians, scientists, and engineers—on what makes a scripting language easy to read and understand in the context of complex mathematical formulations.
        \item Investigate how best to represent mathematical structures in code that is both syntactically correct and user-friendly.
        \item Explore the possibility of using natural language processing to enable easier interaction with the language in later phases.
    \end{itemize}

    \item \textbf{Development of DeduKt Grammar and Syntax:}
    \begin{itemize}
        \item Establish a formal grammar for DeduKt, defining the syntactic rules that will govern the language's structure.
        \item Decide on the core language constructs (e.g., expressions, functions, variables, data types) and how they will be represented within the system.
        \item Address edge cases and ambiguities in mathematical notation, ensuring consistency across the language.
        \item Provide extensibility in the syntax to accommodate future mathematical structures and computational needs.
    \end{itemize}

    \item \textbf{Development of Interpreter, Parser, and Lexer:}
    \begin{itemize}
        \item Build a \textbf{Lexer} to tokenize input scripts, breaking the raw code into manageable components such as numbers, variables, operators, and symbols.
        \item Implement a \textbf{Parser} that will transform tokenized input into an abstract syntax tree (AST), capturing the structure and logic of the mathematical expressions.
        \item Develop the \textbf{Interpreter} to execute mathematical operations and evaluations based on the parsed expressions.
        \item Ensure that the interpreter supports basic arithmetic, algebraic manipulation, and symbolic computation, forming the backbone of DeduKt's functionality.
    \end{itemize}
\end{enumerate}

\subsubsection{Expected Outcomes at the End of Phase 0}
By the end of Phase 0, we expect to have achieved the following outcomes:
\begin{enumerate}
    \item \textbf{Simple Arithmetic with Variables:}
    \begin{itemize}
        \item Users will be able to perform basic arithmetic operations (addition, subtraction, multiplication, division) on variables and constants.
        \item The system will support variables and the ability to manipulate symbolic expressions, forming the groundwork for more complex mathematical operations in later phases.
    \end{itemize}

    \item \textbf{Benchmark on Interpreter and Parser Outcomes:}
    \begin{itemize}
        \item The performance of the interpreter and parser will be benchmarked to assess their efficiency and correctness.
This includes testing the system with various expressions to ensure they are parsed and evaluated as expected.
        \item Initial tests will focus on parsing simple expressions, and later tests will expand to more complex symbolic operations as new features are added.
    \end{itemize}

    \item \textbf{Concrete Syntax Definition and Documentation:}
    \begin{itemize}
        \item A formal, detailed syntax definition will be established and documented, offering clear guidelines for how users can write DeduKt scripts.
        \item This documentation will serve as both a reference for developers and a user manual for future iterations of DeduKt.
        \item Examples of basic expressions and their corresponding syntactic forms will be included to ensure clarity.
    \end{itemize}
\end{enumerate}

\subsubsection{Deliverables of Phase 0}
The deliverables of Phase 0 will include:
\begin{itemize}
    \item A functional version of the interpreter, capable of evaluating simple mathematical expressions and supporting basic algebraic operations.
    \item A formal grammar and syntax specification for DeduKt, including rules for expression evaluation, variable declarations, and basic function definitions.
    \item A set of unit tests for the parser and interpreter to ensure correctness and performance.
    \item Initial user documentation covering the syntax, basic examples, and instructions for running simple computations.
\end{itemize}


    \chapter{References}\label{ch:references}
    \printbibliography

\end{document}